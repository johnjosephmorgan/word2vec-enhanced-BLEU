\section{Introduction}
\label{sec:intro}


\subsection{THE Pocity of References Problem}
\label{sec:pocityprob}

Naturally occurring sentences have billions of correct translations into one foreign language\cite{dreyer-marcu:2012:NAACL-HLT}. 
By a correct translation of a source sentence we mean any sentence in the target language that has the same meaning as the source sentence. 
The BLEU\cite{papineni-02} metric was designed assuming that the system output would be compared to a set of four reference translations. 
In practice system developers rarely have access to more than one  reference translation. 
In this work we assume only one human generated reference is available. 

The designers of BLEU were aware that input sentences could have many translations into a foreign language. 
They incorporated features that capture variations of references. 
Bleu does not require exact match for the entire sentence. 
It allocates credit to sub-sentence segments by considering matches  of overlapping ngrams for   n equals 1 through 4.
Even with this partial credit assignment and assuming four references, it can hardly be claimed that this coverage is enough to be representative of billions of possible correct translations. 
A system can produce a perfectly adequate translation and get a very low BLEU score. 

\subsection{Improving BLEU}
\label{sec:improvebleu}

%BLEU Enhancements
%Outline
%Like bleu we assume the existence of a corpus of human translations to be used for evaluation. 
%We will call this the reference corpus. 
%We enhance BLEU with different methods of generating words or phrases with equivalent meaning to  words or phrases given in the reference corpus. 
Making bleu better is a goal almost as old as BLEU itself. 
Several projects have been devoted to developing metrics that extend BLEU by expanding the set of references to include more sentences with equivalent meaning. 
Some of these projects like meteor\cite{banerjee-lavie:2005:MTSumm}\cite{denkowski:lavie:meteor-wmt:2014}, TERP\cite{snover-06}\cite{snover-08} and HyTER have produced successful metrics. 
It is clear that an automatic method of generating synonyms and sentences with equivalent meaning will improve  BLEU. 


We list some features we would like to see in an improved BLEU metric and some features we want to keep.  
We want to assign credit to approximate matches. 
System output that expresses similar meaning to the input should receive adequate credit from the metric. 
We want  it to adapt easily to a new language pair. 
By easily Adaptable to a new language pair, we mean requiring little or no human annotation. 
We want a metric that assigns reasonable scores to partial output. 
We want to use this metric as a component that measures translation quality in interactive incremental language systems. 
Such systems need to measure the quality of partial inputs. 

%A first step in our project uses a semantic vector space to generate synonyms  automatically for reference words. 
%A second step extends the automatic generation of synonnyms via a semantic vector  space to generate phrases and sentences with similar meaning to a reference sentence by averageing the sequence of vectors in the phrase or sentence. 
%Since averaging  vectors to obtain a vector for a phrase or sentence  is not an adequate model of sentence generation or sentence meaning, a further step uses a compositional model  to produce sentences with equivalent meaning to system output/references. 
%This step leverages  the recent work in   deep learning, specifically the Recursive Neural Tensor Network. 
%The composition of words and phrases in this method is  guided by a parse tree. 

\subsection{Word Similarities}
\label{sec:wordsims}

The ultimate goal of our project is to attack the pocity of references problem directly by using the semantic vector space to generate references equivalent to a single given reference. 
In this paper however, we investigate an easier solution. 
We use the similarity measured defined in the semantic vector space to provide scores for $n$grams that do not have exact matches in the reference. 
Word2vec embedds single word tokens into a vector space, so it makes sense to measure similarity between words in that space. 
The word2vec developers suggest that the similarity function can be extended from words to phrases. 
Although we do not believe that this is theoretically  well-founded, we take on this suggestion and build two enhanced BLEU metrics that compute similarities between phrases. 
One of these metrics yields a significant boost in scores relative to BLEU. 
Many $n$grams that did not match exactly receive a non-zero similarity measure with reference phrases. 
However, an inspection of the matches indicates that a large portion of them are spurious. 
We conclude that this is not an appropriate use of the similarity measure to provide approximate match scores for phrases. 
We instead believe that more sophisticated approaches need to be taken to generate phrases with equivalent meaning to a given reference. 


%Our first two methods use a semantic vector space representation for words.
%A first step uses the similarity function associated with the vector space to score the candidate and reference words. 
%A second step extends the scoring function to phrases by vector averaging. 
%Since averaging  vectors to obtain a vector for a phrase or sentence  is not an adequate model of sentence generation or sentence meaning, a further step uses a compositional model  to produce sentences with equivalent meaning to references. 
%This step leverages  the recent work in   deep learning, specifically the Recursive Neural Tensor Network. 
%The composition of words and phrases in this method is  guided by a parse tree. 

%%% Local Variables: 
%%% mode: latex
%%% TeX-master: "ling848fall2014"
%%% End: 
