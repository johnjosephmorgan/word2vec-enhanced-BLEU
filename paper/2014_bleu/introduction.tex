\section{Introduction}
\label{sec:intro}

Many Army operations involve a need to translate English text into a
foreign language. For example in the case where Army special forces
train foreign army units the texts used in these training operations
should be available in the language of the foreign army.

Depending on its direction, translation can have two purposes for a
country: assimilation, from a foreign language into the country's
native language, or dissemenation, from the country's native language
into a foreign language. Assimilation is highly valued in the
Intelligence Community where analysts want to know what is being
written in foreign texts. Dissemenation of English into low resourced
foreign languages does not receive the attention given to
assimilation. We are interested in using SMT for
dissemenation purposes in Army training operations. Development of
evaluation metrics is an aspect of dissemination that would benefit
from more attention.

The traditional BLEU\cite{BLEU} metric uses exact string matching to score smt
decoder output with reference transcriptions. This makes BLEU
language-independent and thus is equally benefitial for assimilation
and dissemenation purposes. BLEU is well known to have
weaknesses\cite{Callison-Burch2006EACL}. Work has been done to remedy these weaknesses by
incorporating llinguistic features into the evaluation metric. These
enhancements make the metrics more accurate and correlate more with
human judgements.

Meteor\cite{banerjee-lavie2005MTSumm} is an evaluation metric that takes into consideration stem, synonym, and paraphrase matches between words inaddition to exact string matches. Evaluation metrics like meteor exist for resource-rich languages \footnote{Although see: \url{http://www.cs.cmu.edu/~mdenkows/meteor-universal.html} for an interesting development that makes meteor available in other languages. Note that this tool  requires a parallel corpus large enough to train a moses smt system.}, but rare for low-resourced languages of interest to the Army. For dissemmenation purposes, We would like to have an evaluation metric similar to Meteor that does not depend on high powered nlp tools like parsers or pos taggers which are not readily available in most languages.

Meteor extends BLEU by incorporating semantic features into the evaluation of SMT output. We similarly enhance BLEU with semantic knowledge by replacing words in the reference transcriptions with words that are close in meaning. 

The word2vec tool is used to build a vector space word distribution model of the target language. Then a distance function is used to generate a neighborhood of words that are close to the words in the reference transcriptions. Word2vec only has one language-dependent requirement:a large monolingual corpus of text. It does not require a parser, a stemmer, or a part of speech tagger.
The representations of languages are learned using the distributed Skip-gram or Continuous Bag-of-Words (CBOW) models recently proposed by . These models learn word representations using a neural network architecture that   predicts the neighbors of a word.

The traditional BLEU metric measures the overlap of $n$-grams from the decoder
output with $n$-grams from the reference transcription sentences.
Credit is given to the decoder output $n$-gram only when there is an exact
string to string match.
If the reference sentence was:

The boy went to the store

But the decoder output:

The boy went to the supermarket

BLEU would assign no credit to the $1$-gram supermarket.
Although supermarket does not deserve the full creditof $1$ given to store by
BLEU, it seems harsh to give it no credit at all.
Word2vec might place supermarket in a neighborhood of store where the two
words have vectors that have a distance of $0.2$ from each other.
Our algorithm would assign $0.8$ ( $1-0.2$) instead of $0$ to the word supermarket.
We do this until we find a match for a small number ($k=3$) of words in a
neighborhood of the $1$-gram in the reference $1$-gram.
For $n$-grams with $n>1$ the algorithm is slightly more complicated.
We first discard decoder output $n$-grams that differ by more than a single
word.
Then we consider the $n$-grams that result by replacing the word that differs
with words close to it as above. Again
we do this up to $k=3$ times for the top $k$ words that appear in the
neighborhood of the differing word in the reference transcription.
%%% Local Variables: 
%%% mode: latex
%%% TeX-master: "../2014_bleu"
%%% End: 
