
\section{Methods}
\label{sec:methods}

The word2vec\cite{mikalov2013a} tool is used to build a vector space word distribution model of the target language. Then a distance function is used to generate a neighborhood of words that are close to the words in the reference transcriptions. Word2vec only has one language-dependent requirement:a large monolingual corpus of text. It does not require a parser, a stemmer, or a part of speech tagger.
The representations of languages are learned using the distributed Skip-gram or Continuous Bag-of-Words (CBOW) models recently proposed by \cite{mikalov2013a}. These models learn word representations using a neural network architecture that   predicts the neighbors of a word.

The traditional BLEU metric measuresn the overlap of $n$-grams from the decoder
output with $n$-grams from the reference transcription sentences.
Credit is given to the decoder output $n$-gram only when there is an exact
string to string match.
If the reference sentence was:

The boy went to the store

But the decoder output:

The boy went to the supermarket

BLEU would assign no credit to the $1$-gram supermarket.
Although supermarket does not deserve the full creditof $1$ given to store by
BLEU, it seems harsh to give it no credit at all.
Word2vec might place supermarket in a neighborhood of store where the two
words have vectors that have a cosine angle of $0.2$ between each other.
Our algorithm would assign $0.2$ instead of $0$ to the word supermarket.
We do this until we find a match for a small number ($k=3$) of words in a
neighborhood of the $1$-gram in the reference $1$-gram.
For $n$-grams with $n>1$ the algorithm is slightly more complicated.
We first discard decoder output $n$-grams that differ by more than a single
word.
Then we consider the $n$-grams that result by replacing the word that differs
with words close to it as above. Again
we do this up to $k=3$ times for the top $k$ words that appear in the
neighborhood of the differing word in the reference transcription.

%%% Local Variables: 
%%% mode: latex
%%% TeX-master: "../2014_bleu"
%%% End: 
